Containerization is becoming increasingly important in software development to guarantee the security, effectiveness and manageability of containerized applications is crucial. Docker, one of the most popular container platforms relies on Dockerfiles to define the setup and configuration of containers. This means improper or inconsistent Dockerfile practices can lead to bloated,insecure, and inefficient Docker images. This project focuses on developing an automated Docker linter that enforces best practices for Docker file creation, with an emphasis on security,performance and maintainability. 

The Docker checker inspects Dockerfiles for problems like using insecure or outdated base images and having too many layers or unnecessary exposed ports as well as granting root privileges.It also. Ranks "Docker Smells" bad patterns in Dockerfile practices that lead to inefficiency and maintenance problems.The tool offers advice on which issues more urgent and important to address such, as bloated image sizes,long build processes or unnecessary dependencies to ensure a smoother container deployment. The linter recommends utilizing stage builds and optimizing base images to lessen vulnerabilities and trim down image sizes. 

The Docker linter plays a role in CI / CD pipelines by offering immediate feedback to developers to ensure consistency and enhance the quality of their Dockerised environments.This initiative automates the implementation of best practices for Dockerfiles and tackles issues such as "docker smells, this project aims to boost security and performance while making containerized applications easier to maintain and less demanding, on developers.  The new Docker checker is an expandable tool that not just identifies typical errors, in Dockerfiles but also assists companies in upholding strict compliance with their own rules and industry norms.This Docker Linter differs from other tools like Hadolint by continuously updating the best practices by using a web scraper. 